% file:    Commut-Diagram.tex
% date:    July 27, 2005
% author:  Larry Eifler  
%


\typeout{Commutative diagram 1}


   If your dissertation is in pure mathematics, you may
well wonder why graphics and pictures are of concern to you.
Diagrams involving mappings can be made using 
arrays or using the picture facility of LATEX.  

\newcommand{\EEV}{\vector( 1, 0)}
\newcommand{\SSV}{\vector( 0,-1)}
\newcommand{\MB}{\makebox( 0, 0)} 
\newcommand{\SEV}{\vector( 1,-1)}
\newcommand{\SWV}{\vector(-1,-1)}


\begin{equation}
 \begin{array}{c}    % array used to center equation label
 \setlength{\unitlength}{1pt}
 \begin{picture}(125,125)
%
  \put(  0,100){\MB{$A$}}
  \put(100,100){\MB{$B$}}
  \put( 15,100){\EEV{70}}
  \put( 50,110){\MB{$\phi$}} 
% 
  \put(  0, 0){\MB{$\overline{A}$}}
  \put(100, 0){\MB{$\overline{B}$}}
  \put( 15, 0){\EEV{70}}  
  \put( 50,10){\MB{$\overline{\phi}$}} 
% 
  \put(  0, 85){\SSV{70}}
  \put(-10, 50){\MB{$ \pi_A$}} 
%  
  \put(100, 85){\SSV{70}}
  \put(110, 50){\MB{$\pi_B$}} 
%
 \end{picture}
 \end{array}
\end{equation}
\vspace{24pt}

\typeout{Commutative diagram 2}

Even quite complicated diagrams are 
possible using the picture facility. 
After one has coded some simple examples of diagrams,
the simple examples can be used as templates for
ever more complex diagrams.  When coding a new diagram pattern,
first draw the diagram on graph paper.   Then
label the various parts with coordinates. I prefer to work 
with units measured in points so that fine adjustments 
do not lead to fractional units.


\begin{equation}
 \begin{array}{c}    % array used to center equation label
 \setlength{\unitlength}{1pt}
 \begin{picture}(125,75)
%
  \put( 0,50){\MB{$R_i$}}
  \put(15,50){\EEV{70}}
  \put(50,60){\MB{$\phi_j^i$}} 
  \put(15,35){\SEV{20}}
  \put(20,20){\MB{$h_i$}}
%
  \put(100,50){\MB{$R_j$}}
  \put( 85,35){\SWV{20}}
% 
  \put( 50, 0){\MB{$\overline{R}$}}
  \put( 85,20){\MB{$h_j$}}
%
 \end{picture}
\end{array}
\end{equation}
 
\endinput




 