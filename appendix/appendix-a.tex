% file:    Appendix-A-LQE.tex
% date:    July 27, 2005 
% author:  Larry Eifler 
%    

% ==========Appendix A 

\thispagestyle{empty}
\setcounter{chapter}{0}
\def\thechapter{\Alph{chapter}}


%%  This next command can handle bad page break in TOC
%%  \addtocontents{toc}{\protect\addvspace{40pt}}

\addcontentsline{toc}{headingline}{Appendix \hfill \null}
\addappendixline{A USER'S GUIDE FOR THE TEMPLATE}


\null
\vspace{3.25in}
\begin{center}
  APPENDIX A \\
  A USER'S GUIDE FOR THE TEMPLATE
\end{center}
\newpage

\thispagestyle{plain} 

For the most part, the user should be able to 
modify the sources files which come with this template so as to
obtain a properly formatted dissertation. 
Figure \ref{ControlFile} gives a listing of the main control file 
\verb+Main-LQE.tex+.  

\begin{figure}[hbt]
\renewcommand{\arraystretch}{0.65}
\vskip12pt
\small
\begin{center}
\begin{tabular}{|l|}
\hline \\
\verb+  \documentclass{kcmath2e}                     +\\       
\verb+                                               +\\           
\verb+  \usepackage{graphics}                        +\\   
\verb+  \input{Defns-LQE}          % Required        +\\ 
\verb+  \input{Format-LQE}         % Required        +\\ 
\verb+                                               +\\ 
\verb+  \begin{document}                             +\\ 
\verb+                                               +\\ 
\verb+  \include{TitlePage-LQE}                      +\\ 
\verb+  \include{Abstract-LQE}                       +\\ 
\verb+  \include{Acceptance-LQE}                     +\\ 
\verb+  \include{Contents-LQE}                       +\\ 
\verb+  \include{Acknow-LQE}                         +\\ 
\verb+  \include{Dedication-LQE}   % optional        +\\ 
\verb+  \include{EndFront-LQE}                       +\\ 
\verb+  \include{Chap1-LQE}                          +\\ 
\verb+  \include{Chap2-LQE}                          +\\ 
\verb+  \include{Chap3-LQE}                          +\\ 
\verb+  \include{Appendix-A-LQE}                     +\\ 
\verb+  \include{Appendix-B-LQE}                     +\\ 
\verb+  \include{Biblio-LQE}                         +\\ 
\verb+  \include{Vita-LQE}                           +\\ 
\verb+                                               +\\ 
\verb+  \end{document}                               +\\ 
  \\   \hline
\end{tabular}
\end{center}          
\caption{Main control file for template}
\label{ControlFile}
\end{figure}

\clearpage


     A brief description of each of the source files as well as
the style file used to produce this template is  presented below.  

   

\def\filenameheading#1{%
\item {\bf #1 ~~-~~}}  

\begin{enumerate}
\filenameheading{Main-LQE.tex}
This file selects the document style  
and inputs a sequence of files which 
generate the template.  It contains no text of its own.  


\filenameheading{Kcmath2e.cls}
The 12-point book style was used as a basis for this style file.
The \verb+\chapter+ 
and \verb+\section+  macros  are defined so as to satisfy the 
format for dissertations at UMKC\@. Several other details 
concerned with the format requirements specifed 
in the ``Guide''
are handled by this style file.


\filenameheading{Format-LQE.tex}
The margins, spacing and various parameters are defined in this file. 
One may select the depth of the listing for the
table of contents.  One may select if a list of figures or 
a list of tables is to be printed.


\filenameheading{Defns-LQE.tex}
The theorem-like environments are defined here.
Also, macros for starting and ending a proof are defined.  
The blackboard font is defined in this part.



\filenameheading{TitlePage-LQE.tex}
This generates the title page.  
The optional copyright page is also generated.


\filenameheading{Abstract-LQE.tex}
This generates the abstract. 


\filenameheading{Acceptance-LQE.tex}
This generates the acceptance page.  


\filenameheading{Contents.tex}
This file gives commands for generating the table of contents.
If appropriate, a list of figures and/or a list of tables 
is generated. 


\filenameheading{Acknow-LQE.tex}
This generates an acknowledgment page.


\filenameheading{Dedication-LQE.tex}
This generates a dedication page.  
This page is not numbered and is not counted.


\filenameheading{EndFront.tex}
This file generates headings within the table of contents, list of figures
and list of tables.  Also, the numbering of the pages is reset to 1 and
set to arabic style.


\filenameheading{Chap1-LQE.tex}
This contains Chapter 1 of the template.  


\filenameheading{Chap2-LQE.tex}
This  contains  Chapter 2 of the template.  
The \verb+\input+ command can be used to input 
sections or portions of a chapter.
Be careful to distinguish between the commands
\verb+\input+ and \verb+\include+.


\filenameheading{Spring-Mass.tex}
This contains the picture of a spring mass system.


\filenameheading{Commut-Diagram.tex}
This contains two commutative diagrams and a 
short discussion of how they were created.


\filenameheading{Appendix-A-LQE.tex}
This contains Appendix A. You are reading it now!


\filenameheading{Appendix-B-LQE.tex}
This  contains Appendix B.  


\filenameheading{Biblio-LQE.tex}
This file contains the bibliography for the template.


\filenameheading{SampleBiblio.tex}
This document illustrates the AMS-Plain style for a bibliography.


\filenameheading{Vita-LQE.tex}
This file contains the vita page.



\filenameheading{NRheeHex2b.eps}
This is an EPS file.


\filenameheading{YinYang.eps}
This is an EPS file.

\end{enumerate}


% \baselineskip=\normalbaselineskip


\clearpage

\endinput

